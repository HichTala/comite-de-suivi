\begin{table}[h]

    \label{tab:comparison}
    \begin{adjustbox}{width=0.75\textwidth}
        \rowcolors{2}{gray!25}{white}
        % \begin{tabular}{lllTLLT}
        \begin{tabular}{@{}lc}
            \begin{tabular}{@{}lc}
                \toprule[1pt]
                \textbf{Name}                                    & \textbf{Backbone} \\ \hline
                Meta-RCNN~\parencite{wu2020meta}                 & ResNet50          \\
                TFA w/cos~\parencite{wang2020frustratingly}      & ResNet50          \\
                FSCE~\parencite{sun2021fsce}                     & ResNet50          \\
                DeFRCN~\parencite{qiao2021defrcn}                & ResNet50          \\
                FSDiffusionDet~\parentcite{chen2022diffusiondet} & ResNet50          \\
                Distill-cdfsod~\parencite{xiong2023cd}           & ResNet50          \\ \hline
                ViTDeT-FT~\parencite{li2022exploring}            & ViT-B/14          \\ \hline
                Detic~\parencite{zhou2022detecting}              & ViT-L/14          \\
                Detic-FT~\parencite{zhou2022detecting}           & ViT-L/14          \\
                DE-ViT~\parencite{zhang2024detect}               & ViT-L/14          \\
                DE-ViT-FT~\parencite{zhang2024detect}            & ViT-L/14          \\
                CD-ViTO~\parencite{fu2024crossdomain}            & ViT-L/14          \\\bottomrule[1pt]
            \end{tabular}%
        \end{tabular}%
    \end{adjustbox}
    \caption{Liste des méthodes de détection d'objets pouvant être utilisé en cross-domain.
    Ces méthodes seront toutes tester et comparées dans un benchmark à l'aide d'un outil dévoloppé cette année.
    Il sera mis à disposition après la publication du benchmark.}

\end{table}