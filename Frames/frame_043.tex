\begin{subsectionframemod}{Proposed Approaches}
     Determining how much of the model should be fine-tuned is a complex task in FSOD.
     Here, we investigate this choice with thorough experiments on four datasets.

     \begin{table}[]
    \centering
    \resizebox{\columnwidth}{!}{%
    \begin{tabular}{@{\hskip 2mm}lccccc@{\hskip 2mm}}
    \toprule[1pt]
    \textbf{Freezing point}& \textbf{Plasticity}   & \textbf{DOTA} & \textbf{DIOR} & \textbf{Pascal VOC} & \textbf{COCO} \\ \midrule
    FT whole               & 100.00 \%             & 60.09         &  52.17             & 43.10           &  17.15             \\
    Bias only              & 35.98 \%              & \textbf{60.45}&  55.12             & 49.90           &  20.19             \\
    BatchNorm only         & 35.97 \%              & 59.35         &  55.63             & 51.96           &  19.70             \\
    Up to stage 1          & 99.98 \%              & 58.85         &  53.37             & 43.81           &  17.72             \\
    Up to stage 4          & 99.47 \%              & 57.41         &  53.21             & 41.23           &  17.73             \\
    Up to stage 3          & 96.57 \%              & 59.88         &  54.36             & 47.57           &  19.49             \\
    Up to stage 4          & 79.66 \%              & 56.13         &  \textbf{57.51}    & \textbf{53.72}  &  \textbf{21.88}    \\
    FT head only           & 35.97 \%              & 51.82         &  55.70             & 51.72           &  19.96             \\
    FT last layer only     & 0.03 \%               & 0.05          &  0.11              & 0.53            &   0.01             \\ \bottomrule[1pt]
    \end{tabular}%
    }
    \vspace{-0.5em} \caption{Influence of the freezing point on the
    FS performance on DOTA, DIOR, Pascal VOC, and COCO. mAP is reported with a 0.5 IoU threshold and $k=10$ shots.}
    \label{tab:diff_freezing_sweetspot}
    \vspace{-1.5em}
    \end{table}
\end{subsectionframemod}
