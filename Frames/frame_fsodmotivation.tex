\begin{subsectionframemod}{Few-Shot Object Detection}

    \metroset{block=fill}
    \vspace{-10mm}
    \begin{alertblock}{Détection d'objet few-shot}
        La détection d'objet few-shot relève le défi de former des détecteurs d'objets robustes avec un minimum de données étiquetées, ce qui le rend crucial pour les environnements dynamiques et à ressources limitées.
    \end{alertblock}

    Les enjeux sont alors les suivants :
    \begin{itemize}
        \item[-] \textbf{Besoins des applications du monde réel} : Dans de nombreux scénarios réels, le nombre d'images annotées est souvent insuffisant (par exemple, imagerie médicale, images aériennes).
        \item[-] \textbf{Efficacité économique et temporelle} : La détection d'objet few-shot réduit considérablement le \bfalert{temps} et les \bfalert{ressources} nécessaires à la création de grands ensembles de données annotées ainsi que l'entrainement des modèles sur ceux-ci.
    \end{itemize}

\end{subsectionframemod}
